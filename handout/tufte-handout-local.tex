% -------------------------------------------------------------------------- %
% Additional LaTeX packages to build the handout
\usepackage{graphicx}    % allow embedded images
  % Set up graphicx
  \setkeys{Gin}{width=\linewidth,totalheight=\textheight,keepaspectratio}
  \graphicspath{{data/}} % set of paths to search for images
\usepackage{amsmath}     % extended mathematics
\usepackage{booktabs}    % book-quality tables
\usepackage{units}       % non-stacked fractions and better unit spacing
\usepackage{multicol}    % multiple column layout facilities
\usepackage{lipsum}      % filler text
\usepackage{fancyvrb}    % extended verbatim environments
  % Set up fancyvrb
  \fvset{fontsize=\normalsize}% default font size for fancyvrb environments
\usepackage{xspace}      % smart spacing for use at end of macros


% -------------------------------------------------------------------------- %
% Standardize command font styles and environments
\newcommand{\doccmd}[1]{\texttt{\textbackslash#1}}% command name -- adds backslash automatically
\newcommand{\docopt}[1]{\ensuremath{\langle}\textrm{\textit{#1}}\ensuremath{\rangle}}% optional command argument
\newcommand{\docarg}[1]{\textrm{\textit{#1}}}% (required) command argument
\newcommand{\docenv}[1]{\textsf{#1}}% environment name
\newcommand{\docpkg}[1]{\texttt{#1}}% package name
\newcommand{\doccls}[1]{\texttt{#1}}% document class name
\newcommand{\docclsopt}[1]{\texttt{#1}}% document class option name
\newenvironment{docspec}{\begin{quote}\noindent}{\end{quote}}% command specification environment

\renewcommand*\descriptionlabel[1]{\hspace\labelsep\normalfont\textcolor{DarkBlue}{\pyv{#1}}}


% -------------------------------------------------------------------------- %
% Set up colors for URLs
\PassOptionsToPackage{colorlinks=true,
                      urlcolor=DarkBlue,
                      unicode,
                      hyperfootnotes=false,
                      pdfborder = {0 0 0},
                      bookmarksdepth = section,
                      citecolor = DarkGreen,
                      linkcolor = DarkBlue,
                      }{hyperref}
\usepackage{hyperref}


%--------------------------------------------------------------------------- %
% Source code highlighting & customizations with PythonTeX

% Set necessary offset so line numbers do not infringe on left margin
% This assumes the max line number is 4 digits wide
\newlength{\CodeNumberPadding}
\setlength{\CodeNumberPadding}{5pt}
\newlength{\MaxSizeOfLineNumbers}
\settowidth{\MaxSizeOfLineNumbers}{\tiny{9999}}  % Max number of digits
\addtolength{\MaxSizeOfLineNumbers}{\CodeNumberPadding}

% Load the package
\usepackage[gobble=auto]{pythontex}
\usepackage{relsize}  % For relative sizing

% Global PythonTeX typesetting preferences
\setpythontexfv{numbers=left,
                numbersep=\CodeNumberPadding,
                frame=leftline,
                framesep=\CodeNumberPadding,
                rulecolor=\color{gray},
                xleftmargin=\MaxSizeOfLineNumbers,
                fontsize=\normalsize,
                baselinestretch=1.1  % Single space source code
                }

% Set working directory
\setpythontexworkingdir{.}

% Redefine code line numbers to lighten them from black to gray
\renewcommand{\theFancyVerbLine}{\textcolor{gray}{\tiny{\arabic{FancyVerbLine}}}}


% -------------------------------------------------------------------------- %
% End of file
\endinput
